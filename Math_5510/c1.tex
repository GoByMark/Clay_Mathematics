\phantomsection
\section*{1. Reminders on Continuity}
\addcontentsline{toc}{section}{1. Reminders on Continuity}
\phantomsection
\subsection*{1. Continuity}
\addcontentsline{toc}{subsection}{1. Continuity}
\begin{customdefinition}{1.1}[Continuous Function form $\mathbb{R} \longrightarrow \mathbb{R}$]\hypertarget{d_1.1}{\hyperlink{d_3.1}{Click here if you want to know another definition}}
\begin{enumerate}
    \item[1).] A function $f: \mathbb{R} \longrightarrow \mathbb{R}$ is continuous at a point $x \in \mathbb{R}$, if, for all $\epsilon > 0$, there exists a $\delta$ such that 
    $$\left|f(x) - f(y)\right| < \epsilon\,\,\, \text{ whenever }\,\,\, |x- y| < \delta$$
    \item[2).] A function $f: \mathbb{R} \longrightarrow \mathbb{R}$ is continuous if it is continuous at every point $x\in \mathbb{R}$.
\end{enumerate}
\end{customdefinition}

\begin{customdefinition}{1.2}[Euclidean Distance]
Recall that the (Euclidean) distance between two points
$$x = (x_1, \dots, x_n),\,\,\,\,y = (y_1, \dots, y_n)$$
in $\mathbb{R}^n$ is 
$$||x - y|| = \sqrt{(x_1 - y_1)^2 + \dots + (x_n - y_n)^2}$$
\end{customdefinition}

\begin{customdefinition}{1.3}[Continuous Function form $\mathbb{R}^n \longrightarrow \mathbb{R}^m$](So far still good? Well, check this out!)
\begin{enumerate}
    \item[1).] A function $f: \mathbb{R}^n \longrightarrow \mathbb{R}^m$ is continuous at a point $x \in \mathbb{R}^n$, if, for all $\epsilon > 0$, there exists a $\delta > 0$ such that 
    $$\lvert\lvert f(x) - f(y)\rvert\rvert < \epsilon\,\,\, \text{ whenever }\,\,\, \lvert\lvert x - y \rvert\rvert < \delta$$
    \item[2).] A function $f: \mathbb{R}^n \longrightarrow \mathbb{R}^m$ is continuous if it is continuous at every point $x\in \mathbb{R}^n$.
\end{enumerate}
\end{customdefinition}

\begin{customdefinition}{1.4}[Open Ball]
Let $x \in \mathbb{R}^n$ and $\delta > 0$. The open ball at radius $\delta$ centered at x is the set
$$B_\delta (x) = \left\{y \in \mathbb{R}^n \Big| \lvert\lvert x - y\rvert\rvert < \delta\right\}$$
\end{customdefinition}

\begin{customdefinition}{1.5}[Continuity (Reformulated)]
A function $f: \mathbb{R}^n \longrightarrow \mathbb{R}^m$ is continuous at a point $x \in \mathbb{R}^n$, if, for all $\epsilon > 0$, there exists a $\delta > 0$ such that 
    $$f(y) \in B_\epsilon (f(x)) \text{ whenever } y \in B_\delta (x)$$
That is the same as 
    $$f(B_\delta (x)) \subset B_\epsilon (f(x))$$
In words, $f$ is continuous at $x\in \mathbb{R}^n$ if for any $\epsilon > 0$, there exists a $\delta > 0$ such that $f$ sends to open ball of radius $\delta$ at $x$ into the open ball at radius $\epsilon$ at f(x). This sentence gives a precise definition of ``$f$ preserves closeness".\\
In my own words, the image of the open ball of $x$ has to be a subset of the open ball centered at the image of $x$.
\end{customdefinition}

\begin{customdefinition}{1.6}[Open Sets] (Oooohhhhhh! Boy! Here we go!!!!)\\
A subset $U\subset \mathbb{R}^n$ is called open if for all $u \in U$, there exist an $r>0$ such that $$B_r(u) \subset U$$
Open sets, in their generalized form, are fundamental objects in topology.
\end{customdefinition}

\begin{customdefinition}{1.7}[Pre-image] If $f: X \longrightarrow Y$ is a function between sets and $U \subset Y$ is a subset, its pre-image is 
$$f^{-1}(U) = \left\{x\in X \Big| f(x) \in U\right\}$$
\end{customdefinition}

\begin{customthm}{1.1}
A function $f: \mathbb{R}^n \longrightarrow \mathbb{R}^m$ is continuous if and only if, for any open set $U \subset \mathbb{R}^m$, the set $f^{-1}(u) \subset \mathbb{R}^n$ is open.
\end{customthm}

\begin{proof}
First assume that $f$ is continuous and let $U \subset \mathbb{R}^m$ be open. We need to show that $f^{-1}(u) \subset \mathbb{R}^n$ is open. So, let $x \in f^{-1}(u)$, that is 
$$f(x) \in U$$
Since $U$ is open, there exists an $\epsilon > 0$ such that 
$$B_\epsilon(f(x)) \subset U$$
Since $f$ is continuous, there exists a $\delta > 0$ such that 
$$f(B_\delta(x))\subset B_\epsilon(f(x)) \subset U$$
This implies that 
$$B_\delta (x) \subset f^{-1}(U),$$
proving that $f^{-1}(u)$ is open.\\
To prove the converse, suppose that if $U \subset \mathbb{R}^m$ is open, then so too if $f^{-1} (u)$. Let $x \in \mathbb{R}^n$ and $\epsilon > 0$ be given. Then $B_\epsilon(f(x)) \subset \mathbb{R}^m$ is open, so that
$$f^{-1}\left(B_\epsilon(f(x))\right) \subset \mathbb{R}^n$$
is open. Note that $x\in f^{-1}\left(B_\epsilon(f(x))\right)$. So there exists a $\delta > 0$ such that 
$$B_\delta(x) \subset f^{-1} (B_\epsilon (f(x)))$$
that is
$$f(B_\delta(x)) \subset B_\epsilon(f(x))$$
This proves that $f$ is continuous at $x\in\mathbb{R}^n$. Since $x$ was arbitrary, $f$ is continuous.
\end{proof}

\begin{customdefinition}{1.8}[Closed Sets] A subset $A \subset \mathbb{R}^n$ is called closed if its complement
$$A^c = \mathbb{R}^n \setminus A =\left\{x \in \mathbb{R}^n \mid x \notin A\right\}$$
is an open subset.\\
$A$ is closed $\iff$ for each $u\notin A$, there exists an open ball $B_r(u)$ which does not intersect $A$.
\end{customdefinition}
